\documentclass[a4paper, 10pt]{article}
\usepackage{helvet}
\renewcommand{\familydefault}{\sfdefault}
\usepackage{pgf}
\usepackage{eurosym}
\usepackage{graphicx}
\usepackage{wasysym}
\usepackage{hyperref}
\usepackage{listings}
\usepackage{pxfonts}
\usepackage{verbatim}
\usepackage{color}
\usepackage{xcolor}
\usepackage{wrapfig}
\usepackage{enumitem}
\usepackage{booktabs}
\usepackage{gensymb}
\usepackage{tabularx}
\usepackage{currfile}

\hypersetup{
    bookmarks=true,         % show bookmarks bar?
    unicode=true,          % non-Latin characters in Acrobat’s bookmarks
    pdftoolbar=true,        % show Acrobat’s toolbar?
    pdfmenubar=true,        % show Acrobat’s menu?
    pdffitwindow=true,     % window fit to page when opened
    pdftitle={Assessments},    % title
    pdfauthor={Paul Vesey},     % author
    pdfsubject={Building Information Modelling },   % subject of the document
    pdfcreator={},   % creator of the document
    pdfproducer={xelatex}, % producer of the document
    pdfkeywords={'Graphics' }, % list of keywords
    pdfnewwindow=true,      % links in new PDF window
    colorlinks=true,       % false: boxed links; true: colored links
    linkcolor=violet,          % color of internal links (change box color with linkbordercolor)
    citecolor=magenta,        % color of links to bibliography
    filecolor=red,      % color of file links
    urlcolor=blue           % color of external links
}

\setlength\parindent{0pt}
\begin{document}

\lstset{language=HTML,
				basicstyle=\small,
				breaklines=true,
        numbers=left,
        numberstyle=\tiny,
        showstringspaces=false,
        aboveskip=-20pt,
        frame=leftline
        }


\begin{figure}
	\centering
	\includegraphics[width=0.5\linewidth]{./img/TUSlogo}
\end{figure}


\begin{tabularx}{\textwidth}{ |l|X| }
	\hline
	
	\textbf{Subject:} & CADD06021: BIM with Revit Architecture\\
	\textbf{Course:} & Revit Architecture Online\\
	\textbf{Session:} & Spring 2024\\
	\textbf{Lecturer:} & Paul Vesey \footnotesize{BEng, MIE, HDip}\\
	\textbf{Filename:} & \currfilebase\\
	\hline
\end{tabularx}



	
\part*{Assignment 1 – Electrical Systems}

\begin{tabularx}{\textwidth}{ |X|X| }
	\hline
	\textbf{Issue Date:} & 5$^{th}$ October 2021 \\
	\hline 
	\textbf{Submission Date:}  & 23$^{rd}$ October 2021 \\
	\hline
\end{tabularx}


\section*{Continuous Assessment Marks}
This assignment will account for 25\% of the 100\% allocated for continuous assessment in this module

This assignment will examine the following learning outcomes:\\

\begin{tabularx}{\textwidth}{ |c|X|c| }
	\hline
	\textbf{No.} & \textbf{Learning Outcome} & \textbf{Assessed} \\
	\hline 
	1  & Create and analyse Duct layouts in Revit MEP & No \\
	2  & Create and analyse Pipe Layouts in Revit MEP & No \\
	3  & Create and Analyse Electrical Layouts in Revit MEP & Yes \\
	4  & Co-ordinate Mechanical and Electrical Systems in Revit MEP & Yes \\
	\hline
\end{tabularx}





\section*{Assignment Outline}
You will start this assignment by creating a new Revit project based on the electrical template. You will then have to link to the Architectural model for reference.
You are required to model the lighting system and power outlet system as shown in the attached drawing. This will involve placing lights and switches as indicated. Lights and switches are to be allocated to a circuit on the main distribution panel as indicated on the drawing. The details of the light switching arrangements is also shown.
Power outlets are to be placed in the model and allocated to circuits as indicated.
Upon completion of the circuits, you are required to create a panel schedule and re-balance the three-phase load.
You are also required to create an A1 drawing sheet using the LIT title block provided. This sheet should be populated with the views and schedules as shown.

The asset pack for this assignment contains the following items:
\begin{enumerate}
	\item LIT Title-block
	\item Revit Architectural Model
	\item Completed Drawing in pdf format
\end{enumerate}


\section*{Submission}
Upon completion, upload your Revit project file, the architectural model, and a pdf of your drawing to Microsoft Teams on or before the submission deadline.


\section*{Upload Checklist}

\begin{tabular}{|l|l|l|}
	\hline
	\textbf{Item} & \textbf{Format} & \textbf{Filename} \\
	\hline
	Revit File  & Revit Project File & RMEP01-***-00-ZZ-M3-E-001-A1-P01.rvt  \\
	A1 Drawing  & Adobe pdf & RMEP01-***-00-ZZ-DR-E-101-A1-P01.pdf  \\
	\hline
\end{tabular}


\end{document}