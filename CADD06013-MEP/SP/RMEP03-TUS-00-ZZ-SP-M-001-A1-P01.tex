\documentclass[a4paper, 10pt]{article}
\usepackage{helvet}
\renewcommand{\familydefault}{\sfdefault}
\usepackage{pgf}
\usepackage{eurosym}
\usepackage{graphicx}
\usepackage{wasysym}
\usepackage{hyperref}
\usepackage{listings}
\usepackage{pxfonts}
\usepackage{verbatim}
\usepackage{color}
\usepackage{xcolor}
\usepackage{wrapfig}
\usepackage{enumitem}
\usepackage{booktabs}
\usepackage{gensymb}
\usepackage{tabularx}
\usepackage{currfile}

\hypersetup{
    bookmarks=true,         % show bookmarks bar?
    unicode=true,          % non-Latin characters in Acrobat’s bookmarks
    pdftoolbar=true,        % show Acrobat’s toolbar?
    pdfmenubar=true,        % show Acrobat’s menu?
    pdffitwindow=true,     % window fit to page when opened
    pdftitle={Assessments},    % title
    pdfauthor={Paul Vesey},     % author
    pdfsubject={Building Information Modelling },   % subject of the document
    pdfcreator={},   % creator of the document
    pdfproducer={xelatex}, % producer of the document
    pdfkeywords={'Graphics' }, % list of keywords
    pdfnewwindow=true,      % links in new PDF window
    colorlinks=true,       % false: boxed links; true: colored links
    linkcolor=violet,          % color of internal links (change box color with linkbordercolor)
    citecolor=magenta,        % color of links to bibliography
    filecolor=red,      % color of file links
    urlcolor=blue           % color of external links
}

\setlength\parindent{0pt}
\begin{document}

\lstset{language=HTML,
				basicstyle=\small,
				breaklines=true,
        numbers=left,
        numberstyle=\tiny,
        showstringspaces=false,
        aboveskip=-20pt,
        frame=leftline
        }


\begin{figure}
	\centering
	\includegraphics[width=0.5\linewidth]{./img/TUSlogo}
\end{figure}


\begin{tabularx}{\textwidth}{ |l|X| }
	\hline
	
	\textbf{Subject:} & CADD06021: BIM with Revit Architecture\\
	\textbf{Course:} & Revit Architecture Online\\
	\textbf{Session:} & Spring 2024\\
	\textbf{Lecturer:} & Paul Vesey \footnotesize{BEng, MIE, HDip}\\
	\textbf{Filename:} & \currfilebase\\
	\hline
\end{tabularx}

	
\part*{Assignment 3 – Pipe Systems}

\begin{tabularx}{\textwidth}{ |X|X| }
	\hline
	\textbf{Issue Date:} & 5$^{th}$ October 2021 \\
	\hline 
	\textbf{Submission Date:}  & 27$^{th}$ November 2021   \\
	\hline
\end{tabularx}


\section*{Continuous Assessment Marks}
This assignment will account for 25\% of the 100\% allocated for continuous assessment in this module

This assignment will examine the following learning outcomes:\\

\begin{tabularx}{\textwidth}{ |c|X|c| }
	\hline
	\textbf{No.} & \textbf{Learning Outcome} & \textbf{Assessed} \\
	\hline 
	1  & Create and analyse Duct layouts in Revit MEP & No \\
	2  & Create and analyse Pipe Layouts in Revit MEP & Yes \\
	3  & Create and Analyse Electrical Layouts in Revit MEP & No \\
	4  & Co-ordinate Mechanical and Electrical Systems in Revit MEP & Yes \\
	\hline
\end{tabularx}

\vspace{1cm}

\begin{tabularx}{\textwidth}{ |l|X| }
	\hline 
	\textbf{Excellent (70+\%)} & Faithful recreation of the original drawings with no errors, and shows improvements over the original drawing set\\ 
	\hline
	\textbf{Good (56\% to 69\%)} & Recreation of the original drawing set with some minor errors or omissions in presentation and modelling \\
	\hline
	\textbf{Acceptable (40\% to 55\%)} & Recreation of the original drawing set with numerous minor errors or omissions in presentation and modelling that could be addressed with minimal additional work \\ 
	\hline
	\textbf{Poor ($<$40\%)} & Modelling incomplete, Views missing, Major Annotation Missing, general poor presentation of the design  \\
	\hline
\end{tabularx}

\vspace{1cm}





\section*{Assignment Outline}
You will start this assignment by creating a new Revit project based on the piping template. You will then have to link to the Architectural model for reference.  You are required to model three pipe systems, one \textbf{Hydronic Supply} system and one \textbf{Hydronic Return} system and one \textbf{Fire Protection} system as depicted in the drawings that accompany this specification. In order to complete this work you will need to place appropriate sprinklers and radiators as shown on the drawings. The radiator family part has been provided to you. It will be necessary to modify its clearance parameters to effectively model the hydronic systems.  You will also need to create view filters for the Fire System as Revit the standard template does not provide one. You will also have to set the view range parameters as necessary.  You are also required to create three (3) A1 drawing sheets using the LIT title block provided. These sheets should be populated with the views and schedules as shown.  Pipe schedules and are to be placed on the appropriate sheets. All drawings sheets and views are to be replicated in your assignment. You will also need to make use of the tagging functionality in Revit as necessary.\\

The asset pack for this assignment contains the following items:
\begin{enumerate}
	\item LIT Title-block
	\item Revit Architectural Model
	\item Completed Drawings in pdf format
	\item Radiator Family sourced from the NBS
	\item Pipe Accessory Tag Family File
\end{enumerate}


\section*{Submission}
Upon completion, upload your Revit project file, the architectural model, and a pdf of your drawing to Microsoft Teams on or before the submission deadline.

\section*{Upload Checklist}

\begin{tabular}{|l|l|l|}
	\hline
	\textbf{Item} & \textbf{Format} & \textbf{Filename} \\
	\hline
	Revit File  & Revit Project File & RMEP03-***-00-ZZ-M3-M-001-A1-P01.rvt \\
	A1 Drawing  & Adobe pdf & RMEP03-***-00-ZZ-DR-M-101-A1-P01.pdf  \\
	A1 Drawing  & Adobe pdf & RMEP03-***-00-ZZ-DR-M-102-A1-P01.pdf  \\
	A1 Drawing  & Adobe pdf & RMEP03-***-00-ZZ-DR-M-103-A1-P01.pdf  \\
	\hline
\end{tabular}

\end{document}